\section{Einleitung}

\subsection{Gendergerechte Sprache}

Aus Gründen der besseren Lesbarkeit wird auf die gleichzeitige Verwendung der Sprachformen männlich, weiblich und divers (m/w/d) 
verzichtet. Sämtliche Personenbezeichnungen gelten gleichermaßen für alle Geschlechter. 

\subsection{Motivation}

Die sozialen Medien haben sich seit den letzten 10 Jahren rasant verbreitet. Technologien entwickelt sich immer schneller. Besonders im Bereich der Kommunikation
hat sich dort in den letzten Jahren viel getan. Nach der Entwicklung von E-Mails im Jahre 1971
\footnotemark\footnotetext{\Cite{Kroker14.September.2020}, Michael Kroker: Die Evolution von Social Media: Von der ersten E-Mail 
1971 bis 4 Milliarden Nutzer} kamen weitere Meilensteine wie Listserv 1986 und Facebook im Jahre 2004 hinzu. \newline
Doch die große Revolution sozialer Netzwerke, beginnend 2009 mit WhatsApp, ist heute nicht mehr aufzuhalten. Aus dem Alltagsleben kaum noch 
herauszudenken prägen sie sowohl die 
geschäftliche als auch die private Welt. Schon längst sind aktuelle Beispiele wie der Hype der Gamestop Aktie \footnotemark\footnotetext{\Cite{Diedrich23.06.2021}, Sören Diedrich: Gamestop verdient über eine Milliarde Dollar dank Reddit-Hype 
um eigene Aktie}, entstanden durch die 
sozialen Netzwerk,  mehr Norm als noch Einzelfall. Doch wie wirkt sich all das auf die Jüngsten, auf die Jugendlichen, aus? 
Ist es aufgrund der rasanten Entwicklung in den sozialen Medien möglich, Unterschiede bei nur einem geringen 
Altersunterschied festzustellen? Dies und vieles mehr will ich in meiner Facharbeit herausfinden.

\subsection{Leitfrage und Ziel}

Unter meiner Leitfrage \glqq In wie weit unterscheidet sich die Nutzung von sozialen Netzwerken bei Jugendlichen unterschiedlicher
Altersgruppen?\grqq, will ich vor allem den Konsum und die Nutzung sozialer Netzwerke bei Jugendlichen erfassen 
und Differenzen zwischen den Altersgruppen aufzeigen. Dabei sollen vor allem Daten zur Regelmäßigkeit, Dauer, Gründe und Ziele 
verschiedener Nutzer analysiert werden. Es soll eine Tendenz festgehalten werden, die aufzeigt, wie sich dieser Konsum in den
nächsten Jahren entwickeln vorraussichtlich könnte.