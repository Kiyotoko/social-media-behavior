\section{Zusammenfassung}

Jugendliche nutzen die sozialen Medien, um unterhalten zu werden, sich zu informieren und um mit anderen zu schreiben. Diese Gründe können große Auswirkungen darauf haben, welchen Plattformen beliebt sind und welche nicht. Dies zeigt sich zum Beispiel bei der Plattform YouTube.

In der Altersgruppe von 13 bis 18 Jahren sind nach meiner Untersuchung kaum bis keine Unterschiede in Nutzerverhalten festzustellen, außgenommen sind die 17 Jährigen aufgrund des Abiturs. Somit kann man auch bei einem geringen Altersunterschied kaum Unterschiede feststellen.

Erweitert man die Altersspanne der Befragten, würden wahrscheinlich größere Unterschiede sichtbar werden. Dies kann man gut an der Studie von BITKOM und meiner Umfrage erkennen. Beide Ergebnisse beinhalten die Plattform Facebook. Während in der Studie von BITKOM Facebook zu den führenden Plattformen gehört, ist liegt Facebook bei meiner Umfrage auf den letzten Platz.

Im Ergebnis komme ich zu dem Schluss, dass meine Ergebnisse nicht repräsentativ sind. Dies liegt an zu wenigen Teilnehmern, an einer missverständlich gestellten Frage sowie dem Fakt, das die Umfrageteilnehmer weniger einen durchschnittlichen Jugendlichen darstellen als vielmehr einen durchschnittlichen Schüler am FGB.