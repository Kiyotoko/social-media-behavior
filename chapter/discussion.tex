\section{Disskusion}

\subsection{Beantwortung der Leitfrage}

Bei Jugendlichen innerhalb der untersuchten Altersgruppe unterscheidet sich die Nutzung von sozialen Netzwerken kaum. 

Es gibt keine erheblichen Abweichungen in der Beantwortung einzelner Fragen. Festgestellte Anomalien  können nicht vom 
Alter der Person abgeleitet werden. Dies liegt daran, dass
es eine gesamte Altersspanne von nur 6 Jahren gibt. Innerhalb von 6 Jahren sind die Unterschiede hierbei meist nur minimal.

Würde man nun fortlaufend aller 6 Jahre eine solche Umfrage neu durchführen, würde sich im Vergleich zu dieser Umfrage wahrscheinlich 
ein anderes Umfrageergebnisse ergeben. Dies hängt damit zusammen, das nach meiner Analyse meist vorallem die Plattformen genutzt 
werde, die zurzeit in Mode sind. So liegt nach meiner Umfrage Facebook auf dem letzten Platz bei den sozialen Netzwerken, was ein 
komplett anderes Bild ergibt als die Ergebnisse von BITKOM im Jahr 2011.

\subsection{Reflexion}

Ich hätte aufgrund Sars-Cov-19 die Umfrage komplett Online durchführen müssen. Dadurch hätte ich die Möglichkeit gehabt, die Umfrage 
auch dann durchzuführen, wenn Schüler im HomeOffice oder in Quarantäne sind. Zu spät habe ich erkannt, dass mir dadurch ein großer 
Teil meiner Zielgruppe verloren gegangen ist. Daraus resultiert wiederum, dass meine Umfrage weniger repräsentativ ist.

Die Summe der befragten Umfrageteilnehmer ergaben 160 Personen, von denen 133 Personen, 67 männlich, 58 weiblich und 8 diversen 
Geschlechts waren, innerhalb der Zielgruppe fallen. 
Für eine repräsentative Umfrage sind 133 Personen allerdings zu wenig. Um aussagekräftig zu sein, wären
mindestens 500 Personen notwendig gewesen.

Der durchschnittliche finanzielle Status der Eltern eines Schülers am FGB wurde nicht berücksichtigt. Da finanziell besser 
gestellte Eltern ihrem Kind früher ein Gerät kaufen können als Erziehende, bei denen dies nicht so ist, wirkt sich dies auch darauf
aus, ab wann ein Kind erstmals ein solches Geräte nutzen kann. Da am Freien Gymnasium Borsdorf aufgrund des Status als Privatschule 
die Eltern potenziell eher besser verdienen, ist mit einer überdurchschnittlich frühen Nutzung digitaler Geräte zu rechnen. 
Dies stellt allerdings kaum den Durchschnitt der erstmaligen Nutzung dar, weil ausschließlich die Jugendlichen am FGB befragt 
wurden. Um diesen Fakt auszugleichen, hätte man beispielsweise Personen an anderen Schulen befragen können.

Aufgrund der vielen Schulschließungen am FGB hätte ich alle Lehrer schon in den Herbstferien anfragen müssen, um Rechtzeitig alle 
Ergebnisse zu bekommen. Dadurch kam ich in Verzug. Aufgrund einer Sars-CoV-19 Erkrankung musste ich schlussendlich eine
Verlängerung von zwei Wochen beantragen.