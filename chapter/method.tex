\section{Methode}

\subsection{Planung}

Zur Erfassung möglichst vieler Daten habe ich als wissenschaftliche Methode eine Umfrage gewählt. Dab werden zunächst Begriffe 
definiert und abgetrennt, die zu analysierenden Aspekte gesammelt, mögliche Variablen vermerkt und diese auf ihre Messbarkeit geprüft
\footnotemark\footnotetext{\Cite{Konzeption}, Universität Leipzig: Konzeption und Umsetzung}.

Bei der Planung der Umfrage sammelte ich Kriterien, anhand derer man Unterschiede zwischen verschiedenen Altersgruppen erkennen kann.
Der Umfragebogen sollte anfangs nur Online durchzuführen zu sein. Allerdings entschied ich mich dagegen, aufgrund mangelnder Verfügbarkeit
digitaler Geräte in den unteren Klassen.

\subsection{Umfrage}

Innerhalb der Erstellung des Fragebogens überlegte ich mir für jedes Kriterium mehrere passende Fragen. Auf möglich einfache, 
verständliche und eindeutige Fragen wurde geachtet. Dieses überprüfe ich mithilfe einiger Freiwilliger. Dabei bekamen sie eine 
Vorversion des Fragebogens und sollten diesen Ausfüllen. Dadurch konnten zwar Rechtschreibfehler entdeckt werden, allerdings keine 
Probleme in dem Verständnis einzelnen Fragen. Alle daraus gefundenen Fehler wurden korrigiert.

Den Fragebogen erstellte ich sowohl sowohl in Papierform als auch im Onlineformat. Dabei nutzte ich die Plattform Survio.

Die Umfrage wurde von mir auf Datenschutz geprüft und abschließend mit einem entsprechenden Formular an die Schulleitung gesendet.

\subsection{Durchführung}

Nach der Genehmigung des Formulars wurden alle Klassenlehrer kontaktiert. Dabei habe ich nachgefragt, ob ich die Befragung in den
jeweiligen Klassen durchführen kann. Im Falle einer Erlaubnis durch den Klassenlehrer, habe ich  die Anzahl der benötigten 
Umfragebögen ermittelt und diese als auch den Online-Link zur Verfügung gestellt. Die Schüler füllten eigenständig meistens während
einer sozialen Stunde die Umfrage aus. Alle Onlineergebnisse sowie Umfragebögen wurden 
danach in einem einheitlichen digitalen Speicher übertragen.