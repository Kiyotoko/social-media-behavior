\section{Recherche}

\subsection{Abgrenzung von Begriffen}

Soziale Medien dienen einer häufig profilbasierten Vernetzung und Kommunikation von Benutzern über das Internet  \footnotemark
\footnotetext{\Cite{Bendel}, Oliver Bendel: Soziale Medien}. Nicht als soziale Medien gelten Messenger und Suchmaschinen. Damit 
sind Plattformen wie WhatsApp oder Google unzulässig. Unzulässige Plattformen werden während der Umfrage ignoriert.

Als Jugendliche gelten Personen von 13 bis 18 Jahren. Ergebnisse von Teilnehmern, die zwar die Umfrage ausgefüllt haben, aber nicht
in diese Zielgruppe fallen, wurden tendenziell berücksichtigt, hatten allerdings keine Auswirkungen auf die Umfrageergebnisse.

Die Nutzung einer Plattform definiere ich als vom Nutzer gewollte Interaktion mit der jeweils gewählten Plattform. Schaut die Person zum Beispiel auf einer 
Plattform ein Video, so kann das als gewollte Interaktion und somit als Nutzung bezeichnet werde. Wird allerdings Werbung gesehen,
so ist dies nicht gewollt und zählt somit auch nicht als Nutzung.

\subsection{Verhalten von Jugendlichen in sozialen Medien}

Untersuchungen haben ergeben, dass vor allem die Altersgruppe zwischen 16 und 44 Jahren das Internet
nutzen. Auch nutzen Personen von 10 bis 15 sowie von 45 bis 64 Jahren das Internet sehr regelmäßig \footnotemark\footnotetext
{\Cite{11.August.2020}, Statistisches Bundesamt: Durchschnittliche Nutzung des Internets durch Personen nach Altersgruppen}.
Nach aktuellen Ergebnissen sind YouTube, Facebook, Instagram, Pinterest, Twitch und TikTok die führenden Plattformen in Deutschland
\footnotemark\footnotetext{\Cite{Firsching10.November.2021}, Jan Firsching: Social Media Nutzung in Deutschland 2021: Instagram und 
Facebook dominieren \& YouTube bleibt führende Videoplattform }. Diese Ergebnisse spiegeln nur die Beliebtheit bei deutschen 
Nutzern wider. Ich möchte mit meiner Umfrage  Gemeinsamkeiten und Unterschiede von des Nutzerverhaltens von Jugendlichen 
herausarbeiten. 

Eine Studie von BITKOM zeigte 2011 auf, das 12 \% der Nutzer das Handy für soziale Medien verwenden \footnotemark\footnotetext
{\Cite{Meindl20.Febuar.2012}, Claudia Meindl: Infografik soziale Netzwerke. Mit welchen Geräten der Zugriff erfolgt}. Nach meinen 
subjektiven Beobachtungen halte ich, 10 Jahre nach der Untersuchung, diesen Prozentsatz für nicht mehr realistisch. Hierbei
hoffe ich, aktuellere und zuverlässigere Daten herausarbeiten zu können.

\subsection{Methoden der Datenauswertung}

Mit Hilfe der Umfrage soll eine deskriptive Statistik \footnotemark\footnotetext{\Cite{empirisch}, empirio: So wertest du Ergebnisse 
der quantitativen und qualitativen Forschung richtig aus} erstellt werden. Hierbei werden der Median, der arithmetische Mittelwert,
sowie die Korrelation, also der Zusammenhang, ermittelt. Ich will dabei die qualitative Inhaltsanalyse nach Mayring in fünf Schritten 
vollziehen \footnotemark\footnotetext{\Cite{Pfeiffer06.Dezember.2021}, Franziska Pfeiffer: Qualitative Inhaltsanalyse nach Mayring in 
5 Schritten}. 

Dabei wird zuerst das Material ausgewählt, was in meinem Fall der Umfrage entspricht. 

Danach wird die Richtung der Analyse festgelegt. Hierbei untersuche ich eine Zielgruppe, in meinem Fall Jugendliche am Freiem Gymnasium 
Borsdorf. Ich habe die Zielgruppe deshalb ausgewählt, weil sich diese einfach und effektiv untersuchen lässt.

Im dritten Schritt wird die Form der Inhaltsanalyse gewählt. Hier nutze ich die strukturierende Inhaltsanalyse, welche 
gegebenes Material unter vorher festgelegten Kriterien einschätz. Aufgrund der Umfrage wird diese am besten objektiv 
analysiert.

Im vierten Schritt interpretiere ich die Ergebnisse, indem ich die Antworten für die Auswertung auf einfache Zahlen
reduziere. Dabei werden ausschließlich Ordinalskalen \footnotemark\footnotetext{\Cite{Pfeiffer27.Juni.2018}, Franziska Pfeiffer: 
Fragebogen auswerten mit der Häufigkeitsverteilung in Exel - Vorlage und Tipps} verwendet. Das ist damit begründet, dass in der 
Umfrage keine ja / nein Fragen verwendet werden.

Zum Schluss stelle ich die Gütekriterien sicher. Es wird überprüft, ob die Umfrage
transparent, reproduzierbar und objektiv ist.